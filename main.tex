%Document Props
%FONT SIZE
%DOC CLASS
\documentclass[12pt, a4paper]{report}

%Packages
\usepackage{fullpage} %1-inch margins
\usepackage{setspace} %double space
\usepackage[backend=bibtex8]{biblatex}
\addbibresource{ref.bib}
%Change Chapter format in report class
\usepackage{titlesec} 
\titleformat{\chapter}[block]
  {\normalfont\huge\bfseries}{\thechapter.}{1em}{\Huge}
\titlespacing*{\chapter}{0pt}{-19pt}{0pt}

%Graphics package and relative path
\usepackage{graphics}
\graphicspath{{images/}}

% For other font sizes
% !incompatible with fullpage
%\usepackage{extsizes}

\doublespacing
% ----------------------------------------------------------------------------------------------
\begin{document}

% Outer Cover Page
\pagestyle{empty}
\begin{titlepage}
\vspace*{0.2cm}
\begin{center} \textbf{A REPORT\\ON} \end{center}
\begin{center} \textbf{{\Large AUTOMATIC LAND COVER CLASSIFICATION OF MULTI-TEMPORAL SATELLITE IMAGES}} \end{center}
\begin{center} \textbf{BY} \end{center}
\begin{center} 
{\Large 
	\begin{tabular}{c c}
	Guntaas Singh & 2018A7PS0269P\\
	Nisarg Vora & 2018A7PS0254P
	\end{tabular}
}
\end{center}
\begin{center} \textbf{AT} \end{center}
\begin{center} \includegraphics{iirs.png} \end{center}
\begin{center} {\Large Indian Institute of Remote Sensing, Dehradun} \end{center}
\begin{center} A Practice School - I station of \end{center}
\begin{center} {\includegraphics{bits.png}} \end{center}
\begin{center} {\Large Birla Institute of Technology and Science, Pilani} \end{center}
\begin{center} June, 2020 \end{center}
\end{titlepage}
\pagebreak

% Inner Cover Page
\begin{titlepage}
\vspace*{0.3cm}
\begin{center} \textbf{A REPORT\\ON} \end{center}
\begin{center} \textbf{{\Large AUTOMATIC LAND COVER CLASSIFICATION OF MULTI-TEMPORAL SATELLITE IMAGES}} \end{center}
\begin{center} \textbf{BY} \end{center}
\begin{center}
	\begin{tabular}{c c c}
		\textbf{ID Number} & \textbf{Name} & \textbf{Branch} \\
		2018A7PS0269P & Guntaas Singh & B.E. (Hons.) Computer Science \\
		2018A7PS0254P & Nisarg Vora  & B.E. (Hons.) Computer Science \\
	\end{tabular} 
\end{center}
\begin{onehalfspace}
\begin{center} \textbf{Prepared in the partial fulfillment of the} \end{center}
\begin{center} Practice School - I course \end{center}
\begin{center} \textbf{AT} \end{center}
\begin{center} \includegraphics{iirs.png} \end{center}
\begin{center} {\Large Indian Institute of Remote Sensing, Dehradun} \end{center}
\begin{center} A Practice School - I station of \end{center}
\begin{center} {\includegraphics{bits.png}} \end{center}
\begin{center} {\Large Birla Institute of Technology and Science, Pilani} \end{center}
\begin{center} June, 2020 \end{center}
\end{onehalfspace}
\end{titlepage}
\pagebreak

% Acknowledgements
\setcounter{secnumdepth}{0}
\section{Acknowledgements}
\pagestyle{plain}
\pagenumbering{roman}
\setcounter{page}{3}
\paragraph{}
We would sincerely like to thank the Director of Indian Institute of Remote Sensing, Dr. Prakash Chauhan, for giving us an opportunity to work in this organization and gain a significant amount of exposure to corporate work culture, ethics, and etiquettes to be followed while working in a professional environment.
\paragraph{}
We would like to extend our most sincere gratitude to our project in-charge, Dr. Hari Shanker Srivastava, Head of the Programme Planning and Evaluation Group (PPEG) at IIRS, for providing us with the opportunity to work with him on this project, and his guidance and mentorship during the same.
\paragraph{}
We wish to extend our gratitude to the faculty in charge of the PS-I program at IIRS, Dr. Rekha A., Assistant Professor at BITS Pilani - Bangalore Center, for her guidance and advice during the PS-I program, and her helpfulness and responsiveness while addressing all the concerns we raised during the same.
\paragraph{}
In addition, we would also like to thank the members of the Practice School Division, who have worked very hard for operating the PS-I programme remotely to ensure that we have a seamless learning experience.
\pagebreak

% Details and abstract
\begin{center}  
\textbf {BIRLA INSTITUTE OF SCIENCE AND TECHNOLOGY\\
PILANI (RAJASTHAN)\\
Practice School Division}
\end{center}
\begin{onehalfspace}
\textbf{Station:} Indian Institute of Remote Sensing \\
\textbf{Centre:} Dehradun\\
\textbf{Duration:} From 18th May, 2020 to 27th June, 2020 \\
\textbf{Date of start:} 18th May, 2020 \\
\textbf{Date of submission:} 4th June, 2020 \\
\textbf{Title of project:} Automatic Land Cover Classification of Multi-temporal Satellite Images
\begin{center}
\begin{tabular}{c c c}
\textbf{ID Number} & \textbf{Name} & \textbf{Branch} \\
2018A7PS0269P & Guntaas Singh & B.E. (Hons.) Computer Science \\
2018A7PS0254P & Nisarg Vora  & B.E. (Hons.) Computer Science \\
\end{tabular} 
\end{center}
\textbf{Name of guide:} Dr. Hari Shanker Srivastava \\
\textbf{Designation:} Scientist/Engineer - SG. Group Head, Programme Planning and Evaluation Group (PPEG). \\
\textbf{Name of PS faculty:} Dr. Rekha A. 

	\section{Abstract}
	\paragraph{}
	Blah
\end{onehalfspace}
\newpage

%Response sheet
\begin{center}  
\textbf {BIRLA INSTITUTE OF SCIENCE AND TECHNOLOGY\\
PILANI (RAJASTHAN)\\
Practice School Division\\
Response Option Sheet}
\end{center}
\begin{onehalfspace}
\textbf{Station:} Indian Institute of Remote Sensing \\
\textbf{Centre:} Dehradun
\begin{center}
\begin{tabular}{c c c}
\textbf{ID Number} & \textbf{Name} & \textbf{Branch} \\
2018A7PS0269P & Guntaas Singh & B.E. (Hons.) Computer Science \\
2018A7PS0254P & Nisarg Vora  & B.E. (Hons.) Computer Science \\
\end{tabular} 
\end{center}
\textbf{Title of project:} Automatic Land Cover Classification of Multi-temporal Satellite Images\\
\begin{center}
\begin{tabular}{|p{1cm}|p{10cm}|p{3cm}|}
\hline
\textbf{Code No.} & \textbf{Response Option} & \textbf{Course No.(s) and Name} \\
\hline
1 & A new course can be designed out of this project. & ~\\
\hline
2 & The project can help modification of the course content of some of the existing Courses & ~\\
\hline
3 & The project can be used directly in some of the existing Compulsory Discipline Courses (CDC)/ Discipline Courses Other than Compulsory (DCOC)/ Emerging Area (EA), etc. Courses & ~\\
\hline
4 & The project can be used in preparatory courses like Analysis and Application Oriented Courses (AAOC)/ Engineering Science (ES)/ Technical Art (TA) and Core Courses. & ~\\
\hline
5 & This project cannot come under any of the above mentioned options as it relates to the professional work of the host organization. &~ \\
\hline
\end{tabular} 
\end{center}
\end{onehalfspace}
\vspace*{0.5cm}
\textbf{Signature}\\
\textbf{Date: }
\newpage

% Table of Contents
\tableofcontents
\newpage

% INTRODUCTION
\setcounter{page}{1}
\setcounter{secnumdepth}{1}
\chapter{Introduction}
\pagenumbering{arabic}
\section{About IIRS}
\paragraph{}
Formerly known as Indian Photo-interpretation Institute (IPI), the Institute was founded on 21st April 1966 under the aegis of Survey of India (SOI). It was established with the collaboration of the Government of The Netherlands on the pattern of Faculty of Geo-Information Science and Earth Observation (ITC) of the University of Twente, The Netherlands. The original idea of setting the Institute came from India's first Prime Minister Pandit Jawahar Lal Nehru during his visit to The Netherlands in 1957. Since its establishment in 1966, IIRS is a key player for training and capacity building in geospatial technology and its applications through training, education and research in Southeast Asia. The training, education and capacity building programmes of the Institute are designed to meet the requirements of Professionals at working levels, fresh graduates, researchers, academia, and decision makers. IIRS is also one of the most sought after Institute for conducting specially designed courses for the officers from Central and State Government Ministries and stakeholder departments for the effective utilization of Earth Observation (EO) data. Keeping pace with the technological advances, the Institute has enhanced its capability with time, to fulfill the increased responsibility and demand from Indian and international community. Today, it has programmes for all levels of users, i.e. mid-career professionals, researchers, academia, fresh graduates and policy makers. The sustained efforts by its dedicated faculty and the management have made the institute remain in the forefront throughout its journey of about four and a half decades from a photo-interpretation institute to an institute of an international stature in the field of remote sensing and geo-information science.\cite{iirs.about.history}\cite{iirs.about.instiprof}
\section{Remote Sensing}
\paragraph{}
Remote sensing is the acquisition of information about an object or phenomenon without making physical contact with the object and thus in contrast to on-site observation, especially the Earth. Remote sensing is used in numerous fields, including geography, land surveying and most Earth science disciplines (for example, hydrology, ecology, meteorology, oceanography, glaciology, geology); it also has military, intelligence, commercial, economic, planning, and humanitarian applications.  It may be split into "active" remote sensing (when a signal is emitted by a satellite or aircraft to the object and its reflection detected by the sensor) and "passive" remote sensing (when the reflection of sunlight is detected by the sensor). Passive sensors gather radiation that is emitted or reflected by the object or surrounding areas. Reflected sunlight is the most common source of radiation measured by passive sensors. Examples of passive remote sensors include film photography, infrared, charge-coupled devices, and radiometers. Active collection, on the other hand, emits energy in order to scan objects and areas whereupon a sensor then detects and measures the radiation that is reflected or backscattered from the target. RADAR and LiDAR are examples of active remote sensing where the time delay between emission and return is measured, establishing the location, speed and direction of an object. \cite{remotesensingwiki}
\newpage

% ____________________________

% METHODOLOGY
\chapter{Methodology}
\section{Image Classification}
\paragraph{}
Image classification is a standard task in computer vision. In general, the image classification problem involves assigning one  label out of a given fixed set of discrete labels to the input image on the basis of its visual content. While this is a trivial task for humans, robust image classification is a big challenge for a machine. To the computer, the image is just a grid of numbers which entirely change in unreliable ways with variations in viewpoint, illumination, occlusion, etc. As a result, there is no obvious algorithm which solves this problem. However, a data driven approach of providing the machine with many examples of each class and use of machine learning techniques has shown to be useful.\cite{cs231n}\\
There are different ways in which these techniques can be applied for classification of satellite imagery.
\subsection{Pixel Based Approach}
In typical satellite images, pixel sizes are generally similar in size to the objects of interest. Most of the methods for image analysis using remote sensing data work on a per-pixel basis. However, with advances in remote sensing technology, the spatial resolution has become finer than the typical objects of interest, leading to an increase in within-class variablilty.\cite{eyesky}
\subsection{Object Based Approach}
The term "objects" represents meaningful semantic entities or scene components that are distinguishable in an image (e.g., a house, tree or vehicle).\cite{eyesky} This approach involves the partition of the image into meaningful geographical objects that share relatively homogeneous spectral, color, etc.
\subsection{Semantic Approach}
This aims to label each scene image with a specific semantic class.s. Here, a scene image usually refers to a local image patch manually extracted from large scale remote sensing images that contain explicit semantic classes.\cite{eyesky}

\section{Deep Learning and Neural Networks}
Application of traditional machine learning techniques requires handcrafted features, developing which demands a considerable amount of engineering skill and domain expertise. This, however, is not true for neural networks, which automatically learn these features from data using a general-purpose learning procedure.\cite{eyesky, cs231n} Despite having been around for decades, neural networks have garnered much attention only in the last few years on account of the availability of increased computaional power and large amounts of data.
\paragraph{}
A neural network is a computing system made up of a number of simple, highly interconnected processing elements, which process information by their dynamic state response to external inputs.\cite{muruganandham2016semantic}
\newpage


\printbibliography
\end{document}
